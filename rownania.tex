\documentclass{article}
\usepackage[utf8]{inputenc}

\title{kk}
\author{Jakub Koźlak}
\date{June 2021}

\begin{document}

\maketitle


Równanie 1:

$$
\frac{dx}{dt} = (a - by)x
$$
$$
\frac{dy}{dt} = (cx - d)y
$$
a – częstość narodzin ofiar lub współczynnik przyrostu ofiar,\newline
b – częstość umierania ofiar na skutek drapieżnictwa,
\newline
c – częstość narodzin drapieżników lub współczynnik przyrostu drapieżników,\newline
d – częstość umierania drapieżników lub współczynnik ubywania drapieżników,\newline

Równanie 2:

$$
\frac{dx}{dt} = (a - by)x -ex
$$
$$
\frac{dy}{dt} = (cx - d)y-fy
$$

e oraz f to współczynniki konkurencji między osobnikami


Równanie 3:


$$
\frac{dx}{dt} = (a - by)x -gx^2
$$
$$
\frac{dy}{dt} = (cx - d)y-hy^2
$$
g i h to współczynniki umieralności związanej z przepełnieniem obszaru


Równanie 4****

$$
\frac{dx}{dt} = r_1x\left(1- \left(\frac{x+\alpha y}{K_1}\right) \right)
$$
$$
\frac{dy}{dt} = r_2y\left(1- \left(\frac{y+\beta x}{K_2}\right) \right)
$$
$r_1, r_2$ to współczynniki wzrostu populacji\newline
$K_1, K_2$ to pojemnosc srodowiska
$\alpha, \beta$ to wpływ gatunku1 na gatunek2 i odwrotnie

\end{document}
